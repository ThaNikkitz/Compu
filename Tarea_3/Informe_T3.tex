\documentclass[12pt, notitlepage]{article}
\usepackage[margin=1in, top=0.5in]{geometry}
\usepackage[utf8x]{inputenc}
\usepackage{gensymb}
\usepackage{array}
\usepackage{amssymb}
\usepackage{amsmath}
\usepackage{graphicx}
\usepackage{tabularx}
\usepackage{pbox}
\usepackage[makeroom]{cancel}
\usepackage{float}
\usepackage{caption}
\usepackage{newfloat}
\DeclareFloatingEnvironment[name={Gráfico}]{graph}
\newcommand\numberthis{\addtocounter{equation}{1}\tag{\theequation}}

\title{Título}

\date{\today}
\renewcommand\refname{Referencias}
\renewcommand\tablename{Tabla}
\renewcommand\figurename{Figura}
\newcommand{\norm}[1]{\left\lVert#1\right\rVert}

\geometry{letterpaper}

\begin{document}
\thispagestyle{empty}
\setlength{\unitlength}{1 cm} %Especificar unidad de trabajo
\begin{picture}(18,4)
\put(0,0){\includegraphics[scale=0.38]{UTFSM_logo.png}}
\put(11.5,0){\includegraphics[scale=0.2]{mecusm.jpg}}
\end{picture}
\\
\\
\begin{center}
{\LARGE {Universidad Técnica Federico Santa María}}\\[0.5cm]
{\Large Departamento de Ingeniería Mecánica}\\[2cm]
%{\Large Redes}\\[2.3cm]
{\Huge \textbf{Tarea 2: }}\\[0.2cm]
{\Huge \textbf{``Resolución de sistemas de ecuaciones lineales."}}\\[0.2cm]
{\large IPM-458 - Computación Científica.}\\
{\large Alumno: Nicolás Espinoza M.}\\[6cm]
Profesor: Franco Perazzo M.\\
Ayudante: Luis Fuenzalida L.\\[3cm]
Valparaíso - Mayo 12, 2017
\end{center}
\newpage
\tableofcontents
\newpage

\section{Introducción al Problema.}

El desarrollo de este informe gira en torno a un gancho de grúa de sección variable. De la forma de la sección del gancho se conocen algunos puntos en coordenadas $X-Y$, que gracias a métodos numéricos también permitirán definir la periferia de la sección transversal.\\\\
En el trabajo se utilizan Splines Cúbicos que permiten interpolar curvas a los puntos obtenidos en las mediciones presentadas en la tabla entregada en el problema.\\\\
\subsection{Spline cúbico natural.}

Sirve para interpolar puntos mediante un polinomio cúbico. En casos con múltiples puntos, los coeficientes del polinomio varían en cada intervalo. De manera genérica en un dominio $[a,b]$ con intervalos $[x_i,x_{i+1}]$, el polinomio es de la forma
\begin{equation}
S(x_i) = \alpha_i + \beta_i(x-x_i) + \gamma_i(x-x_i)^2 + \delta_i(x-x_i)^3
\end{equation}



\end{document}