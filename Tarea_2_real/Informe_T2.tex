\documentclass[12pt, notitlepage]{article}
\usepackage[margin=1in, top=0.5in]{geometry}
\usepackage[utf8x]{inputenc}
\usepackage{gensymb}
\usepackage{array}
\usepackage{amssymb}
\usepackage{amsmath}
\usepackage{graphicx}
\usepackage{tabularx}
\usepackage{pbox}
\usepackage[makeroom]{cancel}
\usepackage{float}
\usepackage{caption}
\usepackage{newfloat}
\DeclareFloatingEnvironment[name={Gráfico}]{graph}
\newcommand\numberthis{\addtocounter{equation}{1}\tag{\theequation}}

\title{Título}

\date{\today}
\renewcommand\refname{Referencias}
\renewcommand\tablename{Tabla}
\renewcommand\figurename{Imagen}
\newcommand{\norm}[1]{\left\lVert#1\right\rVert}

\geometry{letterpaper}

\begin{document}
\thispagestyle{empty}
\setlength{\unitlength}{1 cm} %Especificar unidad de trabajo
\begin{picture}(18,4)
\put(0,0){\includegraphics[scale=0.38]{UTFSM_logo.png}}
\put(11.5,0){\includegraphics[scale=0.2]{mecusm.jpg}}
\end{picture}
\\
\\
\begin{center}
{\LARGE {Universidad Técnica Federico Santa María}}\\[0.5cm]
{\Large Departamento de Ingeniería Mecánica}\\[2cm]
%{\Large Redes}\\[2.3cm]
{\Huge \textbf{Tarea 2: }}\\[0.2cm]
{\Huge \textbf{``Resolución de sistemas de ecuaciones lineales."}}\\[0.2cm]
{\large IPM-458 - Computación Científica.}\\
{\large Alumno: Nicolás Espinoza M.}\\[6cm]
Profesor: Franco Perazzo M.\\
Ayudante: Luis Fuenzalida L.\\[3cm]
Valparaíso - Mayo 12, 2017
\end{center}
\newpage
\tableofcontents
\newpage

\clearpage
\setcounter{page}{1}

\section{Preguntas no numéricas (1 y 2).}
En general las normas, tanto vectoriales como matriciales, sirven para definir el ``tamaño" de alguno de estos dos objetos. Existen distintas normas que se pueden utilizar. A continuación se explican dos de estas propiedades, relacionándolas con el \textit{número de condición} de una matriz.

\subsection{Factorización LU de una matriz para obtener su inversa.}

Teniendo las matrices $L$ y $U$ en las que se descompone la matriz del sistema $A$, se puede obtener la matriz inversa a través de un paso intermedio. Supongamos que se tienen ya las matrices $L$ y $U$. A continuación se define la operatoria para obtener $A^{-1}$ de manera genérica, para una matriz en este caso $3\text{x}3$ (por simplicidad), aunque el procedimiento se puede aplicar a una matriz $n\text{x}n$ cualquiera:

\begin{equation*}
L = \begin{pmatrix}
l_{11} & 0 & 0 \\
l_{21} & l_{22} & 0 \\
l_{31} & l_{32} & l_{33}
\end{pmatrix} \qquad U=\begin{pmatrix}
u_{11} & u_{12} & u_{13} \\
0 & u_{22} & u_{23} \\
0 & 0 & u_{33}
\end{pmatrix} \qquad A^{-1} = \begin{pmatrix}
a_{11} & a_{12} & a_{13} \\
a_{21} & a_{22} & a_{23} \\
a_{31} & a_{32} & a_{33}
\end{pmatrix}
\end{equation*}
Se define un vector \textbf{D} que contiene elementos para un paso auxiliar que nos ayude a encontrar $A^{-1}$. El planteamiento parte de saber que $A\cdot A^{-1}=I$, y que $L\cdot U = A$.
\begin{gather}
\begin{pmatrix}
l_{11} & l_{12} & l_{13} \\
l_{21} & l_{22} & l_{23} \\
l_{31} & l_{32} & l_{33}
\end{pmatrix} \begin{pmatrix}
d_1 \\ d_2 \\ d_3
\end{pmatrix} = \begin{pmatrix}
1\\0\\0
\end{pmatrix} \quad \rightarrow \quad
\begin{matrix}
l_{11}\cdot d_1 & = 1\\
l_{21}\cdot d_1+ l_{22}\cdot d_2  & = 0 \\
l_{31}\cdot d_1+ l_{32}\cdot d_2 + l_{33}\cdot d_3 & = 0
\end{matrix}
\end{gather}
\begin{equation}
\begin{pmatrix}
u_{11} & u_{12} & u_{13}\\
0 & u_{22} & u_{23}\\
0 & 0 & u_{33}
\end{pmatrix} \begin{pmatrix}
a_{11} \\ a_{21} \\ a_{31}
\end{pmatrix} = \begin{pmatrix}
d_1 \\ d_2 \\ d_3
\end{pmatrix}
\end{equation}

%d_1 = \frac{1}{l_{11}} \quad d_2 = -\frac{l_{21}}{l_{11}\cdot l_{22}} \quad d_3 = \frac{1}{l_{33}}\left(\frac{l_{21}\cdot l_{32}}{l_{11}\cdot l_{22}} - \frac{l_{31}}{l_{11}}\right)

El vector de elementos $a_{ij}$ es la primera columna de la matriz inversa. Repitiendo el mismo procedimiento para las columnas 2 y 3 de la matriz identidad en el sistema (1) hace que el vector \textbf{D} tenga otros valores en sus componentes, que permiten obtener respectivamente las columnas 2 y 3 de la matriz inversa en el sistema (2). Así es que se puede formar la matriz $A^{-1}$ a partir de la descomposición $LU$ de la matriz $A$.

\subsection{Norma de Frobenius.}
La Norma de Frobenius, también conocida como Norma $l_2$, se define como la raíz de la suma de los cuadrados de los elementos de la matriz. Un ejemplo de esto se puede ver en la siguiente matriz $\text{A}_{3x3}$:

\begin{equation*}
A = 
\begin{pmatrix}
a_{11} & a_{12} & a_{13} \\
a_{21} & a_{22} & a_{23} \\
a_{31} & a_{32} & a_{33}
\end{pmatrix} \\
\end{equation*}
\begin{equation*}
||A||_{l_2} = \sqrt{\sum_{i=1}^3 \sum_{j=1}^3 a_{i,j}^2} \quad = \quad \sqrt{a_{1,1}^2 + a_{1,2}^2 + a_{1,3}^2 + a_{2,1}^2 + a_{2,2}^2 + a_{2,3}^2 + a_{3,1}^2 + a_{3,2}^2 + a_{3,3}^2}
\end{equation*}

El caso más fácil de entender esta norma es con un vector. En el caso 2D, la norma $l_2$ nos indica la longitud del vector representado en un plano. Ocurre lo mismo en el caso 3D, aunque ya no es tan fácil de realizar una asociación de este estilo en tres dimensiones.\\

\subsection{Norma Infinito.}

La Norma Infinito también permite visualizar el tamaño de una matriz, pero se trabaja de otra forma: considera sólo el máximo valor de entre las sumas de los elementos de las distintas filas. En el siguiente ejemplo, esta vez con números, se puede ver más claramente esto:

\begin{equation*}
||A||_\infty = \max\limits_{1\leq i \leq 3} \left\lbrace \sum_{j=1}^3 |a_{ij}| \right\rbrace
\end{equation*}

\begin{equation*}
A = \begin{pmatrix}
2 & 5 & 9 \\
1 & 9 & 1 \\
7 & 3 & 0
\end{pmatrix} \quad \rightarrow \quad \max\left\lbrace (2+5+9),(1+9+1),(7+3+0)\right\rbrace = 16
\end{equation*}

\begin{equation*}
\text{Luego, la Norma Infinito de la matriz A del ejemplo sería: } \quad ||A||_\infty = 16
\end{equation*}

Si se define $\textbf{x}$ como la solución real de un sistema $A\textbf{x} = \textbf{b}$, y $\textbf{x}'$ la solución numérica, entonces $\textbf{e}$ es el error en el método numérico, definido como $\mathbf{e} = \mathbf{x} - \mathbf{x}'$.\\
También se define el residual \textbf{r} como $r = \mathbf{r} = A\mathbf{x} - A\mathbf{x}' = \mathbf{b} - \mathbf{b}'$. Luego, la relación de las normas con el \textit{número de condición} de una matriz se define como $cond(A) = \norm{A}\norm{A^{-1}}$, donde $\norm{A}$ es alguna norma de la matriz, por ejemplo una de las anteriormente definidas.\\

El número de condición, y por ende las normas de la matriz, se utiliza para acotar el residual mediante el error. La expresión correspondiente es la siguiente:

\begin{equation*}
\frac{1}{cond(A)} \frac{\norm{\mathbf{r}}}{\norm{\mathbf{b}}} \leq \frac{\norm{\mathbf{e}}}{\norm{\mathbf{x}}} \leq cond(A)\frac{\norm{\mathbf{r}}}{\norm{\mathbf{b}}}
\end{equation*}

\newpage

\section{Programas aplicados para la resolución de una armadura (3 en adelante).}
Los métodos numéricos se pueden utilizar para resolver sistemas de ecuaciones o ecuaciones sin solución analítica de forma comparativamente rápida (en contraste, por ejemplo, con hacerlo ``a mano"), aunque el resultado sólo es aproximado. En la presente sección se utilizarán cuatro formas de resolución distintas para resolver un sistema físico que consiste en una armadura con 5 nodos, sometida a cargas externas, como se ve en la Figura 1. Se explicarán brevementes los métodos a utilizar y se presentarán los resultados correspondientes.

\newpage

\subsection{Método LU.}
El método de Descomposición $LU$, que es el que se utilizó en la sección 2.1, consiste en dividir la matriz de factores $A$ en dos matrices, $L$ una matriz triangular inferior, y $U$ triangular superior que, multiplicadas, den como producto $A$. La idea de hacer esto es resolver dos sistemas más simples en vez del original, que en la mayoría de los casos resulta más complejo. Los sistemas a resolver son:

\begin{equation}
LU\mathbf{x} = \mathbf{b} \quad \longrightarrow \quad \begin{matrix} U\mathbf{x}=\mathbf{y} \\
L \mathbf{y} = \mathbf{b}
\end{matrix}
\end{equation}

Por el método $LU$ se obtienen los siguientes resultados para el valor de las fuerzas de cada barra, así como el valor de las reacciones en el pivote y en el apoyo de la matriz.

\begin{table}[H]
\centering
\caption{Resultados con el método de Descomposición $LU$.}
\begin{tabular}{|c|c|} \hline
$\text{F}_{AB}$ & 548,09888002279502\\ \hline $\text{F}_{BC}$ & 375,12886944524757 \\ \hline $\text{F}_{CD}$ & 424.87113055475254 \\ \hline $\text{F}_{DE}$ & -367,94919248661631 \\ \hline $\text{F}_{BE}$ & 17,586545075681059 \\ \hline $\text{F}_{CE}$ & -14.359353917262496 \\ \hline $\text{F}_{EA}$ & -387.56443472262379  \\ \hline $\text{R}_{A_X}$ & 0.0000000000000000  \\ \hline $\text{R}_{A_Y}$ & 387.56443472262379 \\ \hline $\text{R}_{D_Y}$ & 212,43556527737621\\ \hline
\end{tabular}
\end{table}

El algoritmo computacional lo que hace es recibir como input una matriz $A$ y un vector solución $b$. El método incluye el procedimiento de eliminación Gaussiana, para reducir las matrices $L$ y $U$ a triangulares. Finalmente, el sistema se resuelve a través de la resolución de los sistemas planteados en (3).\\\\
La matriz $A^{-1}$ resultante con el método de descomposición $LU$ es la siguiente:

\begin{equation*}
\begin{pmatrix}
0.000 & 0.000 & 0.328 & 1.086 & 0.328 & 0.568 & 0.000 & 0.000 & 0.000 & 0.758 \\
0.000 & 0.000 & -0.536 & 0.536 & 0.464 & 0.804 & 0.000 & 0.000 & 0.000 & 1.072 \\
0.000 & 0.000 & -0.464 & 0.464 & -0.464 & 1.196 & 0.000 & 0.000 & 0.000 & 0.928 \\
0.000 & 0.000 & 0.402 & -0.402 & 0.402 & -1.036 & 1.000 & 0.000 & 0.000 & -0.804 \\
0.000 & 0.000 & -0.328 & 0.328 & -0.328 & -0.568 & 0.000 & 0.000 & 0.000 & -0.758\\
0.000 & 0.000 & 0.268 & -0.268 & 0.268 & 0.464 & 0.000 & 0.000 & 0.000 & -0.536 \\
0.000 & 0.000 & 0.768 & -0.768 & 0.768 & -0.402 & 1.000 & 0.000 & 1.000 & -0.536 \\
1.000 & 0.000 & 1.000 & 0.000 & 1.000 & 0.000 & 1.000 & 0.000 & 1.000 & 0.000     \\
0.000 & 1.000 & 0.232 & 0.768 & 0.232 & 0.402 & 0.000 & 0.000 & 0.000 & 0.536 \\
0.000 & 0.000 & -0.232 & 0.232 & -0.232 & 0.598 & 0.000 & 1.000 & 0.000 & 0.464
\end{pmatrix}
\end{equation*}

El número de condición correspondiente al problema actual es 16.8, aproximadamente.\\\\

Considerando ahora que la armadura resiste viento en dirección positiva del eje X, se modifica el vector \textbf{b} con los valores correspondientes. Si el viento empuja con una fuerza de 1000 [N], entonces los nodos a la izquierda reciben una fuerza de 500 [N] cada uno. El vector solución es:

\begin{table}[H]
\centering
\caption{Vector solución con viento por la izquierda, en sentido X+.}
\begin{tabular}{|c|c|} \hline
$\text{F}_{AB}$ & 384.01418037868234 \\ \hline $\text{F}_{BC}$ & 643.07806189184680 \\ \hline $\text{F}_{CD}$ & 656.92193810815320 \\ \hline $\text{F}_{DE}$ & -568.91108684657638 \\ \hline $\text{F}_{BE}$ & 181.67124471979366 \\ \hline $\text{F}_{CE}$ & -148.33395009054092 \\ \hline $\text{F}_{EA}$ & -771.53903094592340  \\ \hline $\text{R}_{A_X}$ & -1000.0000000000000   \\ \hline $\text{R}_{A_Y}$ &  271.53903094592340 \\ \hline $\text{R}_{D_Y}$ & 328.46096905407660 \\ \hline
\end{tabular}
\end{table}

Para el caso con viento desde la derecha hacia la izquierda (X-), el vector solución es el siguiente:

\begin{table}[H]
\centering
\caption{Vector solución con viento por la derecha, en sentido X-.}
\begin{tabular}{|c|c|} \hline
$\text{F}_{AB}$ & 712.18357966690758 \\ \hline $\text{F}_{BC}$ & 607.17967699864823  \\ \hline $\text{F}_{CD}$ & 192.82032300135180 \\ \hline $\text{F}_{DE}$ & 333.01270187334381 \\ \hline $\text{F}_{BE}$ & -146.49815456843157 \\ \hline $\text{F}_{CE}$ & 119.61524225601596  \\ \hline $\text{F}_{EA}$ & 496.41016150067588 \\ \hline $\text{R}_{A_X}$ & 1000.0000000000000    \\ \hline $\text{R}_{A_Y}$ &  503.58983849932406 \\ \hline $\text{R}_{D_Y}$ & 96.410161500675912 \\ \hline
\end{tabular}
\end{table}

El caso más desfavorable, como se puede ver por los valores de las fuerzas que actúan sobre las barras, sería en el caso con viento \textbf{hacia} la derecha, pues en la que se encuantra la fuerza de mayor valor, correspondiente a -771.54 [N] sobre la barra EA.

\newpage

\subsection{Método del Pivote de Gauss.}

El Pivote de Gauss, también llamado eliminación Gaussiana, reduce la matriz original a una triangular, de modo de poder resolver el sistema a partir de la resolución de un nuevo sistema más simple, pues la matriz triangular entrega los valores del vector de incognitas en orden. Por ejemplo, si el Pivote de Gauss entrega una diagonal superior, se empieza a resolver desde la ecuación correspondiente a la última fila de la matriz, y así se obtiene inmediatamente una incognita, que después se utiliza en la penúltima ecuación, obteniéndose otro componente del vector solución. Esto se repite hasta llegar a la primera fila.\\
Los resultados de aplicar el Pivote de Gauss al problema número 3 son las fuerzas siguientes:

\begin{table}[H]
\centering
\caption{Valores de las fuerzas según el método del Pivote de Gauss.}
\begin{tabular}{|c|c|} \hline
$\text{F}_{AB}$ & 548.09888002279502 \\ \hline
$\text{F}_{BC}$ & 375.12886944524752 \\ \hline
$\text{F}_{CD}$ & 424.87113055475254 \\ \hline
$\text{F}_{DE}$ & -367.94919248661631 \\ \hline
$\text{F}_{BE}$ & 17.586545075681052 \\ \hline
$\text{F}_{CE}$ & -14.359353917262490 \\ \hline
$\text{F}_{EA}$ & -387.56443472262379 \\ \hline
$\text{R}_{A_X}$ & 0.0000000000000000 \\ \hline
$\text{R}_{A_Y}$ & 387.56443472262379 \\ \hline
$\text{R}_{D_Y}$ & 212.43556527737624 \\ \hline
\end{tabular}
\end{table}

En algunos casos se puede percibir una diferencia entre el método del Pivote y la Descomposición $LU$, pero esto śólo ocurre en el último decimal, si es que ocurre siquiera. El método de Pivote de Gauss es importante, pues es la base del método de Descomposición LU. Se debe recordar que hay que invertir la matriz que surge por realizar el pivoteo para poder despejar el vector de incógnitas.

\newpage

\subsection{Métodos de Gauss-Seidel y Jacobi.}

Para el método de Gauss-Seidel, aplicado a la pregunta número 3, se obtuvieron prácticamente los mismo valores del caso anterior. Hay que hacer notar que, aunque el método se debe aplicar a matrices diagonales dominantes, en este caso eso no es posible, por lo que sólo converge para relajaciones bajo 1.\\\\
Los resultados del sistema, para subrelajación con $\lambda = 0.01$, son los siguientes:

\begin{table}[H]
\centering
\caption{Valores de las fuerzas según el método de Gauss-Seidel.}
\begin{tabular}{|c|c|} \hline
$\text{F}_{AB}$ & 548.09888002279240 \\ \hline
$\text{F}_{BC}$ & 375.12886944524507 \\ \hline
$\text{F}_{CD}$ & 424.87113055474845 \\ \hline
$\text{F}_{DE}$ & -367.94919248661381 \\ \hline
$\text{F}_{BE}$ & 17.586545075677940 \\ \hline
$\text{F}_{CE}$ & -14.359353917260083 \\ \hline
$\text{F}_{EA}$ & -387.56443472262060 \\ \hline
$\text{R}_{A_X}$ & 1.3641074932076653E-012 \\ \hline
$\text{R}_{A_Y}$ & 387.56443472262265 \\ \hline
$\text{R}_{D_Y}$ & 212.43556527737462 \\ \hline
\end{tabular}
\end{table}

Con sobre-relajación (entre 1 y 2), el código entrega sólo NaN, porque el método diverge a causa de la diagonal. El número de iteraciones que se necesitó para cumplir con los requisitos de error es 11578.\\\\
En el caso del método de Jacobi, se tiene el siguiente vector de soluciones para el sistema:

\begin{table}[H]
\centering
\caption{Valores de las fuerzas según el método de Jacobi.}
\begin{tabular}{|c|c|} \hline
$\text{F}_{AB}$ & 548.09888002323316 \\ \hline
$\text{F}_{BC}$ & 375.12886944577542 \\ \hline
$\text{F}_{CD}$ & 424.87113055516045 \\ \hline
$\text{F}_{DE}$ & -367.94919248658721 \\ \hline
$\text{F}_{BE}$ & 17.586545075359265 \\ \hline
$\text{F}_{CE}$ & -14.359353917243100 \\ \hline
$\text{F}_{EA}$ & -387.56443472178842 \\ \hline
$\text{R}_{A_X}$ & 1.1517669589924454E-009 \\ \hline
$\text{R}_{A_Y}$ & 387.56443472284178 \\ \hline
$\text{R}_{D_Y}$ & 212.43556527735947 \\ \hline
\end{tabular}
\end{table}

El número de iteraciones que se necesitó para lograr cumplir con el criterio de error es 8904.\\\\
Se nota que es menor que en el caso de Gauss-Seidel, pero también se puede apreciar que el valor que debería ser cero, $R_{A_X}$, aumentó su valor. Es decir, el método de Jacobi, considerando la diferencia en el número de iteraciones, es menos preciso que el de Gauss-Seidel y ambos son menos precisos que los métodos anteriores.

\newpage

\section{Conclusiones.}

Gauss-Seidel y Jacobi son métodos de resolución por aproximación numérica, por lo que las soluciones son sólo aproximaciones de las soluciones analíticas. Los otros métodos son de resolución directa que, hasta donde alcance la precisión del procesador para trabajar con la data, son soluciones exactas. Es decir, en los casos en los que sea posible, es preferible trabajar con métodos como, por ejemplo, LU. sin embargo, computacionalmente estos métodos pueden resultar muy costosos, por lo que en la mayoría de los casos, las aproximaciones son preferibles.

\newpage

\begin{thebibliography}{3}



\end{thebibliography}

\end{document}
